\documentclass[a4paper,11pt]{article}
\usepackage{amsmath,amsfonts}
\begin{document}
\title{Reading Assignment Ch. 11}
\date{\today}
\author{C. Kimber}
\maketitle
\newpage

\flushleft{2}
\\
\begin{center}
The units of moment of inertia are $kg$ $m^2$. Angular speed
$\omega$ is in units of $rad$ $s^{-1}$. Angular momentum is in
units of $kg$ $m^2$ $s^{-1}$.
\end{center}
\\

5
\\
\begin{center}
An example situation with zero net angular momentum, but with non-zero
rotational and translational angular momentum could involve
a system of particles with rotational angular momentum equal
to the negative of its translational angular momentum about some
reference frame.
\end{center}
\\

6
\\
\begin{center}
Since torque is $\vec{r} \times \vec{F}$, if $\vec{r}$, the radial distance
or $\vec{F}$, the force, are equal to zero, or the force and radial distance are parallel, the
torque about some point will be zero. Alternatively, it is possible
the net torque is zero while having multiple torques about
the point, but where those torques cancel each other out.
\end{center}
\\

8
\\
\begin{center}
Zero net torque.
\end{center}
\\

12
\\
\begin{center}
Bohr's physical interpretation of the hydrogen atom has been
shown to be too simple, but the Bohr radius and magneton are correct and
his model correctly predicts wavelengths of emitted and absorbed
photons for many atoms after accounting for the atomic number of the
nucleus. 
\end{center}
\end{document}
