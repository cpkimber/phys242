\documentclass[a4paper,11pt]{article}
\usepackage{amsmath,amsfonts}
\begin{document}
\title{Reading Assignment Ch. 13}
\date{\today}
\author{C. Kimber}
\maketitle
\newpage

\flushleft{2}
\begin{center}
The crew member will just measure the amount of time it takes for the charge
to reach the end of the meter stick. Using the kinematic equation, and plugging
in zero for the initial velocity and displacements. Then by plugging in the measured time
we can solve for the acceleration.
\\
Once we have the acceleration, we can multiply by the particle mass to get the force.
Knowing the force, we can find the electric field, since the force is just
the product of the charge and the field.
\end{center}
\\

3
\begin{center}
Point charges create electric fields everywhere \it{except for the position that they
occupy.}
\end{center}

\\
6
\begin{center}
Q is the charge of the particle at distance r from the dipole. The dipole has a separation
s (the distance between the positive and negative ends), while q is the magnitude of the
charge of the dipole elements.
\end{center}

\\
10
\begin{center}
10.1 is true. 10.2 is true, and according to the book is the definition of a dipole. 10.3 is
false, dipoles certainly have nonzero electric fields. 10.4 is false, it is proportional to $r^{-3}$.
10.5 is true, this is how fields due to multiple charges are calculated.
\end{center}

\end{document}
