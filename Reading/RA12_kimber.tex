\documentclass[a4paper,11pt]{article}
\usepackage{amsmath,amsfonts}
\begin{document}
\title{Reading Assignment Ch. 12}
\date{\today}
\author{C. Kimber}
\maketitle
\newpage

\flushleft{3}
\\
\begin{center}
The relationship governing this question is
$$\frac{1}{T} = \frac{\partial S}{\partial q}$$
Therefore if the entropy changes rapidly with increasing energy
(a relatively high $\frac{\partial S}{\partial q}$), then
the temperature is low. This inverse proportionality
insinuates high temperatures correspond to small changes
in entropy in relation to energy.
\end{center}

\\
6
\\
\begin{center}
The entropy of an open system can decrease, but the entropy
of the surroundings must increase by a larger amount. The net
change in entropy must be greater than zero according to the
second law of thermodynamics.
\end{center}

\\
7
\\
\begin{center}
I would expect block A to have 60 quanta, and block B to have
40 quanta (1). Additonally, the temperatures would be the same (4),
since their $\frac{dS}{dq}$ would be equal. However, the entropies
would not (3).
\\
True:
$$1, 4$$
\end{center}

\\
10
\\
\begin{center}
It is true that $e^{-\Delta E/k_BT}$ is small when $\Delta E$ is
large (1). It is also true that it is small when T is small (2).
Neither 3 nor 4 are true, however. For example, the ratio of
'spin up spin up' to 'spin up spin down' neutral hydrogen
atoms in the interstellar medium is 3-to-1, so the lowest
energy level is not always the most populated.
\\
True:
$$1, 2$$
\end{center}
\end{document}
